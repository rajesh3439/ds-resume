%----------------------------------------------------------------------------------------
%	PACKAGES AND OTHER DOCUMENT CONFIGURATIONS
%----------------------------------------------------------------------------------------
\documentclass[letterpaper]{DS_class_file} % a4paper for A4

% Command for printing skill overview bubbles
\newcommand\mllibs{ 
~
	\smartdiagram[bubble diagram]{
        \textbf{~~~~~~~Data~~~~~~~},
        \textbf{~~~~~PyTorch~~~~~},
        \textbf{~~~~pyAgrum~~~~},
        \textbf{~~~~causal~~~~}\\\textbf{~~~~-learn~~~~},
        \textbf{~~~~Keras~~~~},
	  \textbf{~~~~~Scikit~~~~~~}\\\textbf{~~~Learn~~~},
	  \textbf{~~TensorFlow~~}
    }
}

\newcommand\pythonlibs{ 
~
	\smartdiagram[bubble diagram]{
        \textbf{~~~~~~Python~~~~~~},
        \textbf{~~~~~Pandas~~~~~},
        \textbf{~~~~~~Numpy~~~~~},
        \textbf{~~~~~PyTest~~~~~},
        \textbf{~~~~~Multi~~~~~}\\\textbf{~~~~Processing~~~~},
        \textbf{~~~~~Requests~~~~},
        \textbf{~~~Streamlit~~~}
    }
}



% \newcommand\hobies{ 
% ~
% 	\smartdiagram[bubble diagram]{
%         \textbf{~~~~Myself~~~~},
%         \textbf{~~~~Reading~~~~}\\\textbf{~~~Books~~~},
%         \textbf{~~~~Travelling~~~~},
%         \textbf{~~~~Watching~~~~}\\\textbf{~~~Movies~~~},
%         \textbf{~~~~Trekking~~~~}
%     }
% }




% Programming skill bars
\programming{
{$\textbullet$ Azure $\textbullet$ Airflow $\textbullet$ Scala / 2.5}, 
{$\textbullet$ MS Office $\textbullet$ Kubernetes $\textbullet$ DevOps / 3.5}, 
{$\textbullet$ Linux $\textbullet$ SQL $\textbullet$ Docker $\textbullet$ Git  / 4.0}, 
{$\textbullet$ Python $\textbullet$ LateX $\textbullet$ Markdown / 6.0}}
\vspace{10mm}
\programminglanguage{{ German (B1 - Intermediate) / 3.5 }, { Kannada (Mother Tongue) / 6}, { Tamil (Mother Tongue) / 6}, { English (Native Language) / 6}}
% Projects text
%\vspace{15mm}
%\languages{
%		\textbf{English-C1}- Advanced in Listening, Reading, Writing and Speaking.\\
%        \textbf{Arabic-A2} - Having basic knowledge about Arabic.
%}

%----------------------------------------------------------------------------------------
%	 PERSONAL INFORMATION
%----------------------------------------------------------------------------------------
% If you don't need onse or more of the below, just remove the content leaving the command, e.g. \cvnumberphone{}
\profilepic{rajesh_cv.png}% My photo is here
\cvname{\hspace{-0.5cm}Rajesh\newline\newline} % My name.
\cvsurname{\hspace{-0.6cm}Rajendran} % My surname
\cvjobtitle{\hspace{0.05cm}Data Scientist} % My position.
% title/career

\cvlinkedin{rajesh-rajendran}
\cvgithub{rajesh3439}
\cvnumberphone{+91 9108291139} % Phone number
\cvsite{Bangalore-India} % Where I live.
\cvwebsite{muhammedbuyukkinaci.com}
\cvmail{rajesh.rajendran@hotmail.co.in} % Email address

% \aboutme{
% I am a T-shaped Data Scientist who has general knowledge \& experience on different IT areas such as Machine Learning Engineering, Back-End \newline Development, Data Engineering, Testing, Amazon Web Services. I possess over 5 years of experience in Python. My passion lies in driving innovation and value through cross-functional collaboration. I adhere to Design Principles while constructing real-world Data Science projects within end-to-end \newline pipelines. I am also in the process of integrating MLOps practices into my work. Beyond these aspects, I am not just an IT Engineer but also an eager lifelong learner.

% Alongside my technical expertise, I exhibit extroversion and an open-minded demeanor, coupled with excellent communication skills. I leverage both my soft and hard skills to establish myself as a leading Data Scientist in the industry.
% }

\aboutme{

\justifying

% I am an aspiring Data Scientist who has experience creating scalable, high-performance and robust data solutions to derive insights from large and complex datasets. I have experience on different areas such as Data Engineering, Machine Learning, DevOps and Unit Testing. My passion is to bring innovation and value through  cross-functional collaboration. Acted as a liaison between stakeholders and technical teams. I have experience recruiting and building high-performance teams.
%  I am an ardent learner who strives for continuous improvement and willing to stretch boundaries to achieve it. 
\begin{itemize}
    \item Constructed scalable, high-performance, robust data solutions to derive insights from large and complex datasets.
    % \item Developed simulation environments for testing wayside signaling devices.
    \item Versatile in working with a wide range of technologies and adept at managing complex tasks.
    \item I am technically proficient and believe the use of technology to solve the business problems. I thrive on challenges and motivated to find innovative solutions.
    \item Acted as a liaison between stakeholders and technical teams.
    \item Successfully recruited and built high-performing teams.
    \item Mentored team members in learning new tools and enhancing existing processes.
\end{itemize}

% \justifying

%  I follow Design Principles to build real-world Data Science projects with end-to-end pipelines. I am also working on integrating MLOps practices into my work. Beyond these aspects, I am not only an IT Engineer but also an eager lifelong learner. 

% \justifying

% Besides my technical expertise, I am an extroverted and open-minded person with excellent communication skills. I leverage both my soft and hard skills to position myself as a leading engineer in the industry.
}

% .

%----------------------------------------------------------------------------------------

\begin{document}

\makeprofile % Printing sidebar

%----------------------------------------------------------------------------------------
%	 Cover Letter
%----------------------------------------------------------------------------------------


%----------------------------------------------------------------------------------------
%	 EXPERIENCE
%----------------------------------------------------------------------------------------

\section{Experience}

\begin{twenty}
    \twentyitem
		{Jun 2022}
		{Present}
		{\hspace{0.3cm}Data Scientist}
		{\href{https://www.alstom.com/alstom-india}{\textbf{Alstom Transport}}}
		{}
		{\begin{itemize}
        \item Developed \textbf{Root Cause analysis} solution utilizing tools from causality on service affecting failures in Urban Transit Systems. Performed \textbf{Causal Discovery} to determine the relationship between variables and failures. Created a \textbf{Causal Bayesian Network} model and exploited the model to derive \textbf{Interventional and Counterfactual} insights about the root causes of failure. Developed and deployed web application to share results with stakeholders. \newline \textit{\textbf{Stack:} Python, causal-learn, causalens, pyArgum, plotly, matplotlib, streamlit, gravis}
        \item Developed anomaly detection algorithm using \textbf{Autoencoder} to identify anomalies in sensor measurements. \newline \textit{\textbf{Stack}: Python, Tensorflow, Keras, Numpy}  
        \item Developed \textbf{analytical application} for deriving insights from datalogs collected from onboard signalling system. Developed data processing modules for computing complex KPIs on huge volume data with \textbf{Big-Data} technology. Developed \textbf{analytical dashboards} and automated end-to-end data pipeline with \textbf{Apache Airflow}. \newline \textit{\textbf{Stack:} Scala, Spark, Minio, Apache Airflow, Postgres, Azure dev-ops, Git, Python, Kubernetes, PowerBI}
        \item Developed a \textbf{statistical algorithm} to identify trains in the fleet that needs \textbf{immediate maintenance} from the event datalogs generated by train. Algorithm is packaged and \textbf{deployed in Alstom's Fleet management system}. \newline \textit{\textbf{Stack}: Python, pyinstaller, Numpy, Pandas}
		\end{itemize}}
		\\
	\twentyitem
		{May 2021}
		{Dec 2023}
		{\hspace{0.3cm}Data Engineer}
		{\href{https://www.alstom.com/alstom-india}{\textbf{Alstom Transport}}}
		{}
		{\begin{itemize}
			\item Designed and implemented \textbf{robust data pipelines} for data injection, data processing and prediction for different data science applications. Incorporate \textbf{data quality checks} into each stage of the data pipeline. Deployed data engineering functions as \textbf{REST API services}.  
   \item \textbf{Optimized data processing} tasks leveraging the Python \textbf{MultiProcessing} libraries. \textbf{Improved processing speed by 10 times}. 
   \item Developed POC for using numpy arrays and python list as \textbf{Pandas alternative for data wrangling} tasks for large volumes of data. \textbf{Reduced processing time by 85\%.}      
   \item \textbf{Optimized PostGres} query performance by addressing \textbf{inefficiencies in database design}. \newline \textit{\textbf{Stack}}: \textit{Python, Azure-devops, Git, Docker, Kubernetes, Flask, shell script, Scala, Spark, Postgres } 
		\end{itemize}}
    \\
	\twentyitem
		{Sep 2015}
		{Jul 2021}
		{\hspace{0.3cm}Software Desinger}
		{\href{https://www.alstom.com/alstom-india}{\textbf{Alstom Transport}}}
		{}
		{\begin{itemize}
			\item Designed and developed \textbf{simulators for Factory Acceptance Testing of Railway Signalling equipment}.
			\item Developed \textbf{track plan creation tool} that allows \textbf{users across globe to collaborate in track plan design and create animations}. \newline \textit{\textbf{Stack}: .Net, MFC, IPC}
		\end{itemize}}
	\\
	\twentyitem
		{Jul 2014}
		{Jul 2015}
		{\hspace{0.3cm}Product Engineering Trainee}
		{\href{}{\textbf{Blue Triangle Innovations}}}
		{}
		{}
		\\
	\twentyitem
    	{Dec 2012}
		{Jun 2014}
        {\hspace{0.3cm} Student Researcher}
        {\textbf{German Aerospace Research Center (DLR)}}
        {}
        {}
\end{twenty}


\newpage

\makeseconda % Print the sidebar

%----------------------------------------------------------------------------------------
%	 EDUCATION
%----------------------------------------------------------------------------------------
\section{Education}

\begin{twenty} % Environment for a list with descriptions

  %       \twentyitem
	 %    {2018 Jan}
	 %    {2020 Jan}
	 %    {\hspace{0.2cm}Master of Science, Industrial Engineering}
	 %    {\href{http://www.ie.boun.edu.tr/}{\hspace{0.27cm} \textbf{Boğaziçi University} }}
	 %    {}
	 %    {\begin{itemize}
		% 	\item Dropped out.
		% \end{itemize}} 
  %       \\
   	\twentyitem
	    {2010 Oct}
	    {2013 Dec}
	    {\hspace{0.2cm}M.Sc in Process Automation}
	    {\href{https://www.tu-dortmund.de/en/}{\hspace{0.27cm} \textbf{Technical University of Dortmund, Germany} }}
	    {}
	    {}
        \twentyitem
	    {2006 Jul}
	    {2010 Apr}
	    {\hspace{0.2cm}B.E in Electronics \& Instrumentation}
	    {\href{https://mitindia.edu/}{\hspace{0.27cm} \textbf{MIT, Anna University} }}
	    {}
	    {} 
\end{twenty}

%----------------------------------------------------------------------------------------
%	 Achievements
%----------------------------------------------------------------------------------------
\section{Achievements}

\begin{twenty} % Environment for a list with descriptions
	\twentyitem
	{Jun 2024}
	{}
	{\hspace{0.3cm}Industrial-Grade Time-Dependent Counterfactual Root Cause Analysis through
the Unanticipated Point of Incipient Failure}
	{}
	{}
	{\begin{itemize}
		\item Paper submitted for \href{https://sites.google.com/view/ci4ts2024/home}{\textbf{"Causal Inference for time-series"}} conference.
        \item Developed a \textbf{Causal Bayesian Network} model based on synthetic data and knowledge graph. Performed \textbf{Counterfactual analysis} and proved root causes are found at the Point of Incipient failure.    
	\end{itemize}}
	\\
	\twentyitem
	{May 2024}
	{}
	{\hspace{0.3cm}\href{https://causalens-foundations-of-causality-certificates.s3.amazonaws.com/a698fab1-0028-4e12-a2a4-5986c7ac6c8d.png} {Certification on Foundations of Causality}}
	{\href{https://causalens.com/}{\textbf{causaLens}}}
	{}
	{
		{\begin{itemize}
				\item Causalens is the leader in causal technology space. This course offered fundamental knowledge and skills for causal AI.
		\end{itemize}}
	}
	\\
    \twentyitem
	{Mar 2024}
	{}
	{\hspace{0.3cm}World Class Expert}
	{\href{https://www.alstom.com/alstom-india}{\textbf{Alstom Transport}}}
	{}
	{\begin{itemize}
			\item Recognized as World Class Expert in developing data solutions.
	\end{itemize}}
    \\
    \twentyitem
	{Jun 2023}
	{}
	{\hspace{0.3cm}\href{https://coursera.org/share/a5ca8400ac0a60d8aeee64f91816d879}{ Mathematics for ML and DS}}
	{\href{https://www.deeplearning.ai/}{\textbf{DeepLearning.Ai}}}
	{}
	{\begin{itemize}
			\item The course covered, core mathematics for machine learning and data science, including linear algebra, calculus, probability, and statistics
	\end{itemize}}
    \\
    \twentyitem
	{\href{}{Aug 2022}}
	{}
	{\hspace{0.3cm}\href{https://coursera.org/share/bd6b7bd9ba1f00e2975568b2f471d170}{Data Science Specialization}}
	{\href{https://www.jhu.edu/}{\textbf{Johns Hopkins University}}}
	{}
	{
    \begin{itemize}
        \item The Data Science Specialization covers the concepts and tools for an entire data science pipeline.
        \item Learned to use the tools, think analytically about complex problems, manage large data sets, deploy statistical principles, create visualizations, build and evaluate machine learning algorithms, publish
reproducible analyses, and develop data products
    \end{itemize}
    }
\end{twenty}


% %----------------------------------------------------------------------------------------
% %	 PROJECTS
% %----------------------------------------------------------------------------------------
% \section{Projects}

% \begin{twenty} % Environment for a list with descriptions
    
% 	\twentyitem
% 	{Django}
% 	{App}
% 	{\hspace{0.3cm}\href{https://muhammedbuyukkinaci.com}{muhammedbuyukkinaci.com}}
% 	{\href{https://github.com/MuhammedBuyukkinaci/muhammedbuyukkinaci.com}{\textbf{GitHub Link}}}
% 	{}
% 	{
% 		{\begin{itemize}
% 				\item A Django Application deployed on a VPS of DigitalOcean.
% 				% \item Used Bootstrap, JavaScript, PostgreSQL, NGINX, Gunicorn and Docker to build up the website.
% 		\end{itemize}}
% 	}
% 	\\
	
% 	\twentyitem
% 	{Time}
% 	{Series}
% 	{\hspace{0.3cm}Bitcoin Trading}
% 	{\href{https://github.com/MuhammedBuyukkinaci/Bitcoin-Trading}{\textbf{GitHub Link}}}
% 	{}
% 	{
% 		{\begin{itemize}
% 				\item Built up an LSTM model to predict price volatility on 4-hourly Bitcoin data.
% 				\item Obtained a fractional signal for trading.
% 		\end{itemize}}
% 	}
% 	\\
	
% 	\twentyitem
% 	{Image}
% 	{Segmentation}
% 	{\hspace{0.3cm}DeText}
% 	{\href{https://github.com/MuhammedBuyukkinaci/DeText}{\textbf{GitHub Link}}}
% 	{}
% 	{
% 		{\begin{itemize}
% 				\item Created a simulated dataset by overlaying text on images.
% 				\item Trained a UNet model to remove text from images via TensorFlow.
% 		\end{itemize}}
% 	}
% 	\\
    
% 	% \twentyitem
% 	% {Image}
% 	% {Classification}
% 	% {\hspace{0.3cm}Multiclass Image Classification With TensorFlow}
% 	% {\href{https://github.com/MuhammedBuyukkinaci/TensorFlow-Multiclass-Image-Classification-using-CNN-s}{\textbf{GitHub Link}}}
% 	% {}
% 	% {
% 	% 	{\begin{itemize}
% 	% 			\item Training a CNN (AlexNet) on multiple classes from scratch.
% 	% 	\end{itemize}}
% 	% }
% % 	\\
% % 	\twentyitem
% % 	{Object}
% % 	{Localization}
% % 	{\hspace{0.3cm}Object Localization With TensorFlow}
% % 	{\href{https://github.com/MuhammedBuyukkinaci/Object-Classification-and-Localization-with-TensorFlow}{\textbf{GitHub Link}}}
% % 	{}
% % 	{
% % 	{\begin{itemize}
% % 			\item Building a multiple-headed CNN, one for regression and one for classification.
% % 	\end{itemize} }
% % 	}
% % 	\twentyitem
% % 	{Sentiment}
% % 	{Analysis}
% % 	{\hspace{0.3cm}Sentiment Analysis with TensorFlow}
% % 	{\href{https://github.com/MuhammedBuyukkinaci/TensorFlow-Sentiment-Analysis-on-Amazon-Reviews-Data}{\textbf{GitHub Link}}}
% % 	{}
% % 	{
% % 		{\begin{itemize}
% % 				\item Solving the binary classification problem with LSTM's and GRU's.
% % 				\item Implementing Conv1D and Conv2D before LSTM and GRU layers.
% % 				\item Adding Attention layer \& BN layer before Dense layers.
% % 		\end{itemize}}
% % 	}
% % 	\\
	
	
% \end{twenty}

% \section{Certificates}

% \begin{twenty} % Environment for a list with descriptions
%         \twentyitem
% 	{AWS}
% 	{}
% 	{\hspace{0.3cm}Introduction to Amazon Web Services}
% 	{\href{https://www.udemy.com/certificate/UC-8773cc22-c86d-43a9-89ad-4c4e8e02aa38/}{\textbf{Certificate Link}}}
% 	{}
% 	{
% 		{\begin{itemize}
% 				\item Completed a 25-hours AWS Course on Udemy.
% 				\item Learned fundamentals of AWS and services like EC2, S3, IAM, VPC, RDS, ECR, EKS and DynamoDB etc.
% 		\end{itemize}}
% 	}
% 	\\

% 	\twentyitem
% 	{Docker}
% 	{}
% 	{\hspace{0.3cm}Docker A-Z™}
% 	{\href{https://www.udemy.com/certificate/UC-c1ab98de-9803-452b-9166-8ef3ae797e5a/}{\textbf{Certificate Link}}}
% 	{}
% 	{
% 		{\begin{itemize}
% 				\item Completed a 16-hours course on Udemy. Learned basics of Docker.
% 		\end{itemize}}
% 	}
%          \\
%         \twentyitem
% 	{Kubernetes}
% 	{}
% 	{\hspace{0.3cm}Kubernetes Basics}
% 	{\href{https://www.udemy.com/certificate/UC-ffd4189d-0bd4-4cde-9845-bf7c5e5bbf22/}{\textbf{Certificate Link}}}
% 	{}
% 	{
% 		{\begin{itemize}
% 				\item Completed a 17-hours Kubernetes Course on Udemy.
%                     \item Learned fundamentals of Kubernetes and K8s objects such as Pod, Deployment, Service, Ingress, Cronjob etc.
% 		\end{itemize}}
% 	}
% 	\\
% 	\twentyitem
% 	{MLOps}
% 	{}
% 	{\hspace{0.3cm}Complete MLOPS Bootcamp}
% 	{\href{https://www.udemy.com/certificate/UC-2296034a-bb96-4d35-80b5-dd956cceeeb7/}{\textbf{Certificate Link}}}
% 	{}
% 	{
% 		{\begin{itemize}
% 				\item Completed a 3.5 hours MLOps course.
%                     \item Learned useful tools for MLOps practices.
% 		\end{itemize}}
% 	}
% 	\\
% 	\twentyitem
% 	{Linux}
% 	{}
% 	{\hspace{0.3cm}Linux A-Z™}
% 	{\href{https://www.udemy.com/certificate/UC-74719d94-89af-44ba-ab2d-4299dd2ec3dd/}{\textbf{Certificate Link}}}
% 	{}
% 	{
% 		{\begin{itemize}
% 				\item Completed a 10-hours Linux Course on Udemy.
% 				\item Elevated my Linux knowledge to an advanced level.
% 		\end{itemize}}
% 	}
% 	\\
% 	\twentyitem
% 	{Big Data}
% 	{}
% 	{\hspace{0.3cm}Big Data A-Z™}
% 	{\href{https://www.udemy.com/certificate/UC-4a851250-cc99-4a9d-8f7c-d8a32ed0a832/}{\textbf{Certificate Link}}}
% 	{}
% 	{
% 		{\begin{itemize}
% 				\item Completed a 12-hours Big Data Course on Udemy.
% 				\item Learned basics of Apache Kafka, Apache Hive, Hadoop, Apache Spark and other NoSQL technologies.
% 		\end{itemize}}
% 	}
% 	\\
% 	\twentyitem
% 	{Airflow}
% 	{}
% 	{\hspace{0.3cm}Introduction to Apache Airflow}
% 	{\href{https://www.udemy.com/certificate/UC-634f3164-fcb1-4bdf-b5b0-909134dd3252/}{\textbf{Certificate Link}}}
% 	{}
% 	{
% 		{\begin{itemize}
% 				\item Completed a 2.5 hours Airflow course.
%                     \item Learned fundamentals of Airflow.
% 		\end{itemize}}
% 	}
	
% \end{twenty}

% \section{References}

% \begin{twenty} % Environment for a list with descriptions
% 	\twentyitem
% 	{Reference 1}
% 	{}
% 	{\hspace{0.3cm}Onur Tüfekçioğlu}
% 	{\textbf{Senior System Engineer at Turkcell}}
% 	{}
% 	{
% 		{\begin{itemize}
% 				\item E-mail Address: tufekciogluonur@gmail.com
% 		\end{itemize}}
% 	}
% 	\\
% 	\twentyitem
% 	{Reference 2}
% 	{}
% 	{\hspace{0.3cm}Fatih Öztürk}
% 	{\textbf{Data Scientist at h2o.ai}}
% 	{}
% 	{
% 		{\begin{itemize}
% 				\item E-mail Address: fatihozturk1994@gmail.com
% 		\end{itemize}}
% 	}

	
% \end{twenty}



\end{document}

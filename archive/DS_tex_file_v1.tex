%----------------------------------------------------------------------------------------
%	PACKAGES AND OTHER DOCUMENT CONFIGURATIONS
%----------------------------------------------------------------------------------------
\documentclass[letterpaper]{DS_class_file} % a4paper for A4

% Command for printing skill overview bubbles
\newcommand\mllibs{ 
~
	\smartdiagram[bubble diagram]{
        \textbf{~~~~~~~Data~~~~~~~},
        \textbf{~~~~~PyTorch~~~~~},
        \textbf{~~~~pyAgrum~~~~},
        \textbf{~~~~causal~~~~}\\\textbf{~~~~-learn~~~~},
        \textbf{~~~~Keras~~~~},
	  \textbf{~~~~~Scikit~~~~~~}\\\textbf{~~~Learn~~~},
	  \textbf{~~TensorFlow~~}
    }
}

\newcommand\pythonlibs{ 
~
	\smartdiagram[bubble diagram]{
        \textbf{~~~~~~Python~~~~~~},
        \textbf{~~~~~Pandas~~~~~},
        \textbf{~~~~~~Numpy~~~~~},
        \textbf{~~~~~PyTest~~~~~},
        \textbf{~~~~~Multi~~~~~}\\\textbf{~~~~Processing~~~~},
        \textbf{~~~~~Requests~~~~},
        \textbf{~~~Streamlit~~~}
    }
}



% \newcommand\hobies{ 
% ~
% 	\smartdiagram[bubble diagram]{
%         \textbf{~~~~Myself~~~~},
%         \textbf{~~~~Reading~~~~}\\\textbf{~~~Books~~~},
%         \textbf{~~~~Travelling~~~~},
%         \textbf{~~~~Watching~~~~}\\\textbf{~~~Movies~~~},
%         \textbf{~~~~Trekking~~~~}
%     }
% }




% Programming skill bars
\programming{
{$\textbullet$ Azure $\textbullet$ Airflow $\textbullet$ Scala / 2.5}, 
{$\textbullet$ MS Office $\textbullet$ Kubernetes $\textbullet$ DevOps / 3.5}, 
{$\textbullet$ Linux $\textbullet$ SQL $\textbullet$ Docker $\textbullet$ Git  / 4.0}, 
{$\textbullet$ Python $\textbullet$ LateX $\textbullet$ Markdown / 6.0}}
\vspace{10mm}
\programminglanguage{{ German (B1 - Intermediate) / 3.5 }, { Kannada (Mother Tongue) / 6}, { Tamil (Mother Tongue) / 6}, { English (Native Language) / 6}}
% Projects text
%\vspace{15mm}
%\languages{
%		\textbf{English-C1}- Advanced in Listening, Reading, Writing and Speaking.\\
%        \textbf{Arabic-A2} - Having basic knowledge about Arabic.
%}

%----------------------------------------------------------------------------------------
%	 PERSONAL INFORMATION
%----------------------------------------------------------------------------------------
% If you don't need onse or more of the below, just remove the content leaving the command, e.g. \cvnumberphone{}
\profilepic{rajesh_cv.png}% My photo is here
\cvname{\hspace{-0.5cm}Rajesh\newline\newline} % My name.
\cvsurname{\hspace{-0.6cm}Rajendran} % My surname
\cvjobtitle{\hspace{0.05cm}Data Scientist} % My position.
% title/career

\cvlinkedin{rajesh-rajendran}
\cvgithub{rajesh3439}
\cvnumberphone{+91 9108291139} % Phone number
\cvsite{Bangalore-India} % Where I live.
% \cvwebsite{rajesh3439}
\cvmail{rajesh.rajendran@hotmail.co.in} % Email address

% \aboutme{
% I am a T-shaped Data Scientist who has general knowledge \& experience on different IT areas such as Machine Learning Engineering, Back-End \newline Development, Data Engineering, Testing, Amazon Web Services. I possess over 5 years of experience in Python. My passion lies in driving innovation and value through cross-functional collaboration. I adhere to Design Principles while constructing real-world Data Science projects within end-to-end \newline pipelines. I am also in the process of integrating MLOps practices into my work. Beyond these aspects, I am not just an IT Engineer but also an eager lifelong learner.

% Alongside my technical expertise, I exhibit extroversion and an open-minded demeanor, coupled with excellent communication skills. I leverage both my soft and hard skills to establish myself as a leading Data Scientist in the industry.
% }

\aboutme{

\justifying

As a Data Science professional, I excel in all stages of data science projects, from ideation to production. My passion lies in driving innovation and creating value through cross-functional collaboration. I am dedicated to continuous improvement and am always eager to push boundaries to achieve excellence.
% Data Science professional with experience working in all stages of data science projects from Ideation to Production. My passion is to bring innovation and value through  cross-functional collaboration. I am an ardent learner who strives for continuous improvement and willing to stretch boundaries to achieve it. 

%scalable, high-performance and robust data solutions to derive insights from large and complex datasets. I have experience on different areas such as Data Engineering, Machine Learning, DevOps and Unit Testing. My passion is to bring innovation and value through  cross-functional collaboration. Acted as a liaison between stakeholders and technical teams. I have experience recruiting and building high-performance teams.

% DataScience professional having experince creating end-end data science projects from ideation, proeject definition, methodology, development, deployment.


%  I am an ardent learner who strives for continuous improvement and willing to stretch boundaries to achieve it. 

% \begin{itemize}
%     \item I am technically proficient and believe in the use of technology to solve the business problems. I thrive on challenges and motivated to find novel solutions.
%     \item Built scalable, high-performance, robust data solutions to derive insights from large and complex datasets.
%     % \item Developed simulation environments for testing wayside signaling devices.
%     \item Versatile in working with a wide range of technologies and adept at managing complex tasks.
%     \item Acted as a liaison between stakeholders and technical teams.
%     \item Successfully recruited and built high-performing teams.
%     \item Mentored team members in learning new tools and enhancing existing processes.
% \end{itemize}

% \justifying

%  I follow Design Principles to build real-world Data Science projects with end-to-end pipelines. I am also working on integrating MLOps practices into my work. Beyond these aspects, I am not only an IT Engineer but also an eager lifelong learner. 

% \justifying

% Besides my technical expertise, I am an extroverted and open-minded person with excellent communication skills. I leverage both my soft and hard skills to position myself as a leading engineer in the industry.
}
\competencies{
\justifying
\begin{itemize}
    \item Data Science
    \item Machine Learning
    \item Data Engineering
    \item Software Engineering
    \item Team Building
\end{itemize}
}

% .

%----------------------------------------------------------------------------------------

\begin{document}

\makeprofile % Printing sidebar

%----------------------------------------------------------------------------------------
%	 Cover Letter
%----------------------------------------------------------------------------------------

%----------------------------------------------------------------------------------------
%	 EXPERIENCE
%----------------------------------------------------------------------------------------

\section{\faCogs Experience}

\begin{twenty}
    \twentyitem
        {May 2023}
		{Present}
		{Senior Data Scientist}
		{\href{https://www.alstom.com/alstom-india}{\textbf{Alstom Transport}}}
		{}
		{\underline{\textbf{Project: Service-Affecting Failure Root Cause Analysis}}
        \vspace{0.1cm} 
        \newline
        A solution to identify the root-cause of service affecting failures in Urban Transit Systems for effective troubleshooting. %utilizing tools from \textbf{causality}.
        \begin{itemize}
        \item \textbf{As Data Scientist} 
        \begin{itemize}
            % \item Defined Data Science methodology for the solution.
            \item Performed \textbf{Causal Discovery} on data-logs to obtain a relationship graph, revealing the underlying causal structure between events and failures.
            \item Developed a \textbf{Causal model} and applied \textbf{Interventional and Counterfactual} techniques to identify root causes of failure.
        \end{itemize}
        \item \textbf{As Data Engineer} 
        \begin{itemize}
            \item Created and deployed end-end \textbf{Data pipeline} for \textbf{Data Ingestion, Data Processing (Cleaning, Transformation), and Data Visualization}.
            \item Created and deployed a \textbf{web-based pedagogical tool} to share results with stakeholders.
        \end{itemize}
        \textit{\textbf{Tools \& Technology:} Python, pyAgrum, causalearn, matplotlib, plotly, D3.js, Kubernetes, Azure DevOps, Minio, PostGres, Azure Blob storage}
        \end{itemize}
        }
    \\
    \twentyitem
		{Jun 2021}
		{May 2023}
		{Junior Data Scientist}
		{\href{https://www.alstom.com/alstom-india}{\textbf{Alstom Transport}}}
		{}
		{
        \underline{\textbf{Project: Log Analytics}}
        \vspace{0.1cm} 
        \newline
        An Analytical solution for detecting undesirable system behaviour from data-logs. 
        % Onboard signaling system troubleshooters spend weeks analyzing hundreds of data logs, each containing millions of records, to identify system health deterioration or detect failures. Developed an analytical application utilizes this data to provide train health indicator metrics.
        \begin{itemize}
        \item \textbf{As Data Scientist}
            \begin{itemize}
                \item Worked with system experts to define the \textbf{Health Indicators} to detect the undesirable behaviours. Design rules to \textbf{detect erroneous patterns} in the logs 
                \item Developed \textbf{PowerBI} dashboards application to help troubleshooters identify unhealthy patterns appearing in data-logs and efficiently navigate through them. Thus reducing the troubleshooting time by 80\%
            \end{itemize}
        \item \textbf{As Data Engineer}
            \begin{itemize}
                \item Developed analytics modules for computing complex health indicators using \textbf{Big-Data} technology
                \item Automated the end-to-end data pipeline (\textbf{Data Injection, Data Processing (Cleaning, Transformation)}) with \textbf{Apache Airflow}
            \end{itemize}
        \textit{\textbf{Tools \& Technology:} Scala, Spark, Minio, Apache Airflow, Postgres, Azure DevOps, Git, Python, Kubernetes, PowerBI}
        \end{itemize}
        \vspace{0.3cm}
        % \newline 
        \underline{\textbf{Project: Incident Detection}}
        \vspace{0.1cm} 
        \newline
        A Data-driven troubleshooting solution to detect unhealthy trains from a fleet that needs maintenance. 
        \begin{itemize}
            \item Developed a \textbf{statistical model} to detect faulty trains from the event data-logs generated by train. 
            \item Packaged solution as executable and \textbf{deployed in Alstom's Fleet management system}. 
        \end{itemize}
        \textit{\textbf{Tools \& Technology:} Python, pyinstaller, Numpy, Pandas}
        \vspace{0.3cm}
        \newline 
        \underline{\textbf{Project: Anomaly detection in Odometry system}}
        \vspace{0.1cm} 
        \newline
        A solution to detect anomalies in the train's odometry system to ensure efficient and effective maintenance.
        % Based on data captured by onboard odometry system in train, we developed a solution to detect anomalies in the train's axle speed and ensure efficient and effective maintenance.
        \begin{itemize}
            \item Developed algorithm using \textbf{Autoencoder} architecture that detected anomalies with high Recall (over 99\%)
		\end{itemize}
        \textit{\textbf{Tools \& Technology}: Python, Autoencoder, Tensorflow, Keras, Numpy, Dash}
		}
		\\
	% \twentyitem
	% 	{May 2021}
	% 	{Dec 2023}
	% 	{\hspace{0.3cm}Data Engineer}
	% 	{\href{https://www.alstom.com/alstom-india}{\textbf{Alstom Transport}}}
	% 	{}
	% 	{
 %        \begin{itemize}
	% 		\item Designed and implemented \textbf{robust data pipelines} for data injection, data processing and prediction for different data science applications. Incorporate \textbf{data quality checks} into each stage of the data pipeline. Deployed data engineering functions as \textbf{REST API services}.  
 %   \item \textbf{Optimized data processing} tasks leveraging the Python \textbf{MultiProcessing} libraries. \textbf{Improved processing speed by 10 times}. 
 %   \item Developed POC for using numpy arrays and python list as \textbf{Pandas alternative for data wrangling} tasks for large volumes of data. \textbf{Reduced processing time by 85\%.}      
 %   \item \textbf{Optimized PostGres} query performance by addressing \textbf{inefficiencies in database design}. \newline \textit{\textbf{Stack}}: \textit{Python, Azure-devops, Git, Docker, Kubernetes, Flask, shell script, Scala, Spark, Postgres } 
	% 	\end{itemize}}
 %    \\
	\twentyitem
		{Sep 2015}
		{Jul 2021}
		{Software Designer}
		{\href{https://www.alstom.com/alstom-india}{\textbf{Alstom Transport}}}
		{}
		{ Worked on design and development of \textbf{tools and simulators} for Factory Acceptance Testing of Railway Signalling equipment
  %       \begin{itemize}
		% 	\item Designed and developed \textbf{simulators for Factory Acceptance Testing of Railway Signalling equipment}.
		% 	\item Developed \textbf{track plan creation tool} that allows \textbf{users across globe to collaborate in track plan design and create animations}. \newline \textit{\textbf{Stack}: .Net, MFC, IPC}
		% \end{itemize}
        }
	\\
	\twentyitem
		{Jul 2014}
		{Jul 2015}
		{Product Engineering Trainee}
		{\href{}{\textbf{Blue Triangle Innovations}}}
		{}
		{}
	% 	\\
	% \twentyitem
 %    	{Dec 2012}
	% 	{Jun 2014}
 %        {\hspace{0.3cm} Student Researcher}
 %        {\textbf{German Aerospace Research Center (DLR)}}
 %        {}
 %        {}
\end{twenty}


\newpage

\makeseconda % Print the sidebar

%----------------------------------------------------------------------------------------
%	 EDUCATION
%----------------------------------------------------------------------------------------
\section{\faUniversity Education}

\begin{twenty} % Environment for a list with descriptions

  %       \twentyitem
	 %    {2018 Jan}
	 %    {2020 Jan}
	 %    {\hspace{0.2cm}Master of Science, Industrial Engineering}
	 %    {\href{http://www.ie.boun.edu.tr/}{\hspace{0.27cm} \textbf{Boğaziçi University} }}
	 %    {}
	 %    {\begin{itemize}
		% 	\item Dropped out.
		% \end{itemize}} 
  %       \\
   	\twentyitem
	    {2010 Oct}
	    {2013 Dec}
	    {\hspace{0.2cm}M.Sc in Process Automation}
	    {\href{https://www.tu-dortmund.de/en/}{\hspace{0.27cm} \textbf{Technical University of Dortmund, Germany} }}
	    {}
	    {}
        \twentyitem
	    {2006 Jul}
	    {2010 Apr}
	    {\hspace{0.2cm}B.E in Electronics \& Instrumentation}
	    {\href{https://mitindia.edu/}{\hspace{0.27cm} \textbf{MIT, Anna University} }}
	    {}
	    {} 
\end{twenty}

%----------------------------------------------------------------------------------------
%	 CERTIFICATIONS
%----------------------------------------------------------------------------------------

\section{\faCertificate Certificates}

\begin{twenty} % Environment for a list with descriptions
	\twentyitem
    {ML}
	{}
	{\hspace{0.3cm} Machine Learning Specialization}
    {\href{https://coursera.org/share/ff6e7764bffa429050b673b4cad74ae5} {\textbf{Certificate Link}}}
	% {\href{https://www.deeplearning.ai/}{\textbf{DeepLearning.Ai}}}
	{}
	{
		{\begin{itemize}
            \item Supervised Machine Learning: Regression (\textbf{Linear Regression}) and Classification (\textbf{Logistic Regression})
            \item Advanced Learning Algorithms: \textbf{Multi Layer Perceptron (MLP) Neural Networks, Decision Trees, Random Forest, XGBoost} 
            \item Unsupervised Learning, \textbf{Recommenders (Collaborative filtering, content-based filtering), Reinforcement Learning}
		\end{itemize}}
	}
	\\
	\twentyitem
	{Causal AI}
	{}
    {\hspace{0.3cm} Foundations of Causality}
	{\href{https://causalens-foundations-of-causality-certificates.s3.amazonaws.com/a698fab1-0028-4e12-a2a4-5986c7ac6c8d.png} {\textbf{Certificate Link}}}
	% {\href{https://causalens.com/}{\textbf{causaLens}}}
	{}
	{
		{\begin{itemize}
				% \item This course offered fundamental knowledge and skills for causal AI.
            \item Introduction to \textbf{cutting-edge and widely researched} topic of Causal AI
            \item \textbf{Causal Discovery, Causal Modelling, Root Cause Analysis (Interventional, Counterfactual), Algorithmic Recourse} 
		\end{itemize}}
	}
	\\
    \twentyitem
	{Math for}
	{ML and DS}
    {\hspace{0.3cm} Mathematics insights on Data Science}
	{\href{https://coursera.org/share/a5ca8400ac0a60d8aeee64f91816d879}{\textbf{Certificate Link}}}
	% {\href{https://www.deeplearning.ai/}{\textbf{DeepLearning.Ai}}}
	{}
	{\begin{itemize}
			\item The course covered, core mathematics for machine learning and data science, including \textbf{linear algebra, calculus, probability, and statistics}
	\end{itemize}}
    \\
    \twentyitem
	{DS}
	{}
	{\hspace{0.3cm} Data Science Specialization}
    {\href{https://coursera.org/share/bd6b7bd9ba1f00e2975568b2f471d170}{\textbf{Certificate Link}}}
	% {\href{https://www.jhu.edu/}{\textbf{Johns Hopkins University}}}
	{}
	{
    \begin{itemize}
        \item The specialization imparts skills and tools necessary for developing end-end \textbf{Data Science Pipeline}. It constitutes of \textbf{10 courses} covering core concepts of \textbf{Data Engineering, Machine Learning, Visualizations} and developing \textbf{Data Products}.   
        % \item Learned to use the tools, think analytically about complex problems, manage large data sets, deploy statistical principles, create visualizations, build and evaluate machine learning algorithms, publish reproducible analyses, and develop data products
    \end{itemize}
    }
	% \twentyitem

\end{twenty}


%----------------------------------------------------------------------------------------
%	 Achievements
%----------------------------------------------------------------------------------------
\section{\faTrophy Achievements}

\begin{twenty} % Environment for a list with descriptions
	\twentyitem
	{Jun 2024}
	{}
    {\hspace{0.3cm} Paper submitted on international conference}
	{}
	{}
	{\begin{itemize}
        \item Core contributor in paper submitted on \textbf{CausalAI} on topic \textbf{Industrial-Grade Time-Dependent Counterfactual Root Cause Analysis through the Unanticipated Point of Incipient Failure} to the conference \href{https://sites.google.com/view/ci4ts2024/home}{\textbf{"Causal Inference for time-series"}} held in Barcelona.
		% \item Paper submitted for \href{https://sites.google.com/view/ci4ts2024/home}{\textbf{"Causal Inference for time-series"}} conference.
        % \item Developed a \textbf{Causal Bayesian Network} model based on synthetic data and knowledge graph. Performed \textbf{Counterfactual analysis} and proved root causes are found at the Point of Incipient failure.    
	\end{itemize}}
	\\
    \twentyitem
	{Mar 2024}
	{}
	{\hspace{0.3cm}World Class Expert}
	{\href{https://www.alstom.com/alstom-india}{\textbf{Alstom Transport}}}
	{}
	{\begin{itemize}
			\item Recognized as World Class Expert in developing data solutions.
	\end{itemize}}
    \\
    \twentyitem
	{Aug 2023}
	{}
	{\hspace{0.3cm}Winner SoftWar 2.0}
	{\href{https://www.alstom.com/alstom-india}{\textbf{Alstom Transport}}}
	{}
	{\begin{itemize}
			\item Won the competition by applying AI for estimating tuning parameters of Odometry system.
	\end{itemize}}
    
\end{twenty}


% %----------------------------------------------------------------------------------------
% %	 PROJECTS
% %----------------------------------------------------------------------------------------
% \section{Projects}

% \begin{twenty} % Environment for a list with descriptions
    
% 	\twentyitem
% 	{Django}
% 	{App}
% 	{\hspace{0.3cm}\href{https://muhammedbuyukkinaci.com}{muhammedbuyukkinaci.com}}
% 	{\href{https://github.com/MuhammedBuyukkinaci/muhammedbuyukkinaci.com}{\textbf{GitHub Link}}}
% 	{}
% 	{
% 		{\begin{itemize}
% 				\item A Django Application deployed on a VPS of DigitalOcean.
% 				% \item Used Bootstrap, JavaScript, PostgreSQL, NGINX, Gunicorn and Docker to build up the website.
% 		\end{itemize}}
% 	}
% 	\\
	
% 	\twentyitem
% 	{Time}
% 	{Series}
% 	{\hspace{0.3cm}Bitcoin Trading}
% 	{\href{https://github.com/MuhammedBuyukkinaci/Bitcoin-Trading}{\textbf{GitHub Link}}}
% 	{}
% 	{
% 		{\begin{itemize}
% 				\item Built up an LSTM model to predict price volatility on 4-hourly Bitcoin data.
% 				\item Obtained a fractional signal for trading.
% 		\end{itemize}}
% 	}
% 	\\
	
% 	\twentyitem
% 	{Image}
% 	{Segmentation}
% 	{\hspace{0.3cm}DeText}
% 	{\href{https://github.com/MuhammedBuyukkinaci/DeText}{\textbf{GitHub Link}}}
% 	{}
% 	{
% 		{\begin{itemize}
% 				\item Created a simulated dataset by overlaying text on images.
% 				\item Trained a UNet model to remove text from images via TensorFlow.
% 		\end{itemize}}
% 	}
% 	\\
    
% 	% \twentyitem
% 	% {Image}
% 	% {Classification}
% 	% {\hspace{0.3cm}Multiclass Image Classification With TensorFlow}
% 	% {\href{https://github.com/MuhammedBuyukkinaci/TensorFlow-Multiclass-Image-Classification-using-CNN-s}{\textbf{GitHub Link}}}
% 	% {}
% 	% {
% 	% 	{\begin{itemize}
% 	% 			\item Training a CNN (AlexNet) on multiple classes from scratch.
% 	% 	\end{itemize}}
% 	% }
% % 	\\
% % 	\twentyitem
% % 	{Object}
% % 	{Localization}
% % 	{\hspace{0.3cm}Object Localization With TensorFlow}
% % 	{\href{https://github.com/MuhammedBuyukkinaci/Object-Classification-and-Localization-with-TensorFlow}{\textbf{GitHub Link}}}
% % 	{}
% % 	{
% % 	{\begin{itemize}
% % 			\item Building a multiple-headed CNN, one for regression and one for classification.
% % 	\end{itemize} }
% % 	}
% % 	\twentyitem
% % 	{Sentiment}
% % 	{Analysis}
% % 	{\hspace{0.3cm}Sentiment Analysis with TensorFlow}
% % 	{\href{https://github.com/MuhammedBuyukkinaci/TensorFlow-Sentiment-Analysis-on-Amazon-Reviews-Data}{\textbf{GitHub Link}}}
% % 	{}
% % 	{
% % 		{\begin{itemize}
% % 				\item Solving the binary classification problem with LSTM's and GRU's.
% % 				\item Implementing Conv1D and Conv2D before LSTM and GRU layers.
% % 				\item Adding Attention layer \& BN layer before Dense layers.
% % 		\end{itemize}}
% % 	}
% % 	\\
	
	
% \end{twenty}

% %----------------------------------------------------------------------------------------
% %	 CERTIFICATIONS
% %----------------------------------------------------------------------------------------

% \section{Certificates}

% \begin{twenty} % Environment for a list with descriptions
%         \twentyitem
% 	{AWS}
% 	{}
% 	{\hspace{0.3cm}Introduction to Amazon Web Services}
% 	{\href{https://www.udemy.com/certificate/UC-8773cc22-c86d-43a9-89ad-4c4e8e02aa38/}{\textbf{Certificate Link}}}
% 	{}
% 	{
% 		{\begin{itemize}
% 				\item Completed a 25-hours AWS Course on Udemy.
% 				\item Learned fundamentals of AWS and services like EC2, S3, IAM, VPC, RDS, ECR, EKS and DynamoDB etc.
% 		\end{itemize}}
% 	}
% 	\\

% 	\twentyitem
% 	{Docker}
% 	{}
% 	{\hspace{0.3cm}Docker A-Z™}
% 	{\href{https://www.udemy.com/certificate/UC-c1ab98de-9803-452b-9166-8ef3ae797e5a/}{\textbf{Certificate Link}}}
% 	{}
% 	{
% 		{\begin{itemize}
% 				\item Completed a 16-hours course on Udemy. Learned basics of Docker.
% 		\end{itemize}}
% 	}
%          \\
%         \twentyitem
% 	{Kubernetes}
% 	{}
% 	{\hspace{0.3cm}Kubernetes Basics}
% 	{\href{https://www.udemy.com/certificate/UC-ffd4189d-0bd4-4cde-9845-bf7c5e5bbf22/}{\textbf{Certificate Link}}}
% 	{}
% 	{
% 		{\begin{itemize}
% 				\item Completed a 17-hours Kubernetes Course on Udemy.
%                     \item Learned fundamentals of Kubernetes and K8s objects such as Pod, Deployment, Service, Ingress, Cronjob etc.
% 		\end{itemize}}
% 	}
% 	\\
% 	\twentyitem
% 	{MLOps}
% 	{}
% 	{\hspace{0.3cm}Complete MLOPS Bootcamp}
% 	{\href{https://www.udemy.com/certificate/UC-2296034a-bb96-4d35-80b5-dd956cceeeb7/}{\textbf{Certificate Link}}}
% 	{}
% 	{
% 		{\begin{itemize}
% 				\item Completed a 3.5 hours MLOps course.
%                     \item Learned useful tools for MLOps practices.
% 		\end{itemize}}
% 	}
% 	\\
% 	\twentyitem
% 	{Linux}
% 	{}
% 	{\hspace{0.3cm}Linux A-Z™}
% 	{\href{https://www.udemy.com/certificate/UC-74719d94-89af-44ba-ab2d-4299dd2ec3dd/}{\textbf{Certificate Link}}}
% 	{}
% 	{
% 		{\begin{itemize}
% 				\item Completed a 10-hours Linux Course on Udemy.
% 				\item Elevated my Linux knowledge to an advanced level.
% 		\end{itemize}}
% 	}
% 	\\
% 	\twentyitem
% 	{Big Data}
% 	{}
% 	{\hspace{0.3cm}Big Data A-Z™}
% 	{\href{https://www.udemy.com/certificate/UC-4a851250-cc99-4a9d-8f7c-d8a32ed0a832/}{\textbf{Certificate Link}}}
% 	{}
% 	{
% 		{\begin{itemize}
% 				\item Completed a 12-hours Big Data Course on Udemy.
% 				\item Learned basics of Apache Kafka, Apache Hive, Hadoop, Apache Spark and other NoSQL technologies.
% 		\end{itemize}}
% 	}
% 	\\
% 	\twentyitem
% 	{Airflow}
% 	{}
% 	{\hspace{0.3cm}Introduction to Apache Airflow}
% 	{\href{https://www.udemy.com/certificate/UC-634f3164-fcb1-4bdf-b5b0-909134dd3252/}{\textbf{Certificate Link}}}
% 	{}
% 	{
% 		{\begin{itemize}
% 				\item Completed a 2.5 hours Airflow course.
%                     \item Learned fundamentals of Airflow.
% 		\end{itemize}}
% 	}
	
% \end{twenty}

% \section{References}

% \begin{twenty} % Environment for a list with descriptions
% 	\twentyitem
% 	{Reference 1}
% 	{}
% 	{\hspace{0.3cm}Onur Tüfekçioğlu}
% 	{\textbf{Senior System Engineer at Turkcell}}
% 	{}
% 	{
% 		{\begin{itemize}
% 				\item E-mail Address: tufekciogluonur@gmail.com
% 		\end{itemize}}
% 	}
% 	\\
% 	\twentyitem
% 	{Reference 2}
% 	{}
% 	{\hspace{0.3cm}Fatih Öztürk}
% 	{\textbf{Data Scientist at h2o.ai}}
% 	{}
% 	{
% 		{\begin{itemize}
% 				\item E-mail Address: fatihozturk1994@gmail.com
% 		\end{itemize}}
% 	}

	
% \end{twenty}



\end{document}
